\documentclass{beamer}
\usetheme{Madrid}
\usecolortheme{default}

% Packages
\usepackage[utf8]{inputenc}
\usepackage{graphicx}
\usepackage{tikz}
\usetikzlibrary{shapes, arrows, positioning}

% Title page information
\title{Introduction to DBMS}
\subtitle{Understanding the Foundation of DBMS}
\author{Nafiul Islam}

\date{\today}

\begin{document}

% Title slide
\begin{frame}
    \titlepage
\end{frame}

% Table of contents
\begin{frame}{Outline}
    \tableofcontents
\end{frame}

% Section 1
\section{Introduction to Databases}

\begin{frame}{What is a Database?}
    \begin{definition}
        A \textbf{database} is an organized collection of structured data stored electronically in a computer system, designed to be easily accessed, managed, and updated.
    \end{definition}
    
    \vspace{0.5cm}
    
    \begin{block}{Key Characteristics}
        \begin{itemize}
            \item Organized and structured data storage
            \item Persistent storage of information
            \item Efficient data retrieval and manipulation
            \item Support for multiple users and applications
        \end{itemize}
    \end{block}
\end{frame}

\begin{frame}{Evolution of Data Storage}
    \begin{columns}
        \column{0.5\textwidth}
        \textbf{Traditional Approach:}
        \begin{itemize}
            \item Paper-based records
            \item File cabinets
            \item Manual indexing
            \item Limited accessibility
        \end{itemize}
        
        \column{0.5\textwidth}
        \textbf{Modern Database Systems:}
        \begin{itemize}
            \item Digital storage
            \item Automated indexing
            \item Concurrent access
            \item Real-time updates
        \end{itemize}
    \end{columns}
\end{frame}

% Section 2
\section{Relational Databases}

\begin{frame}{Relational Database Concept}
    \begin{block}{Definition}
        A \textbf{relational database} organizes data into tables (relations) consisting of rows and columns, where relationships between data items are established through common fields.
    \end{block}
    
    \vspace{0.5cm}
    
    \textbf{Key Components:}
    \begin{itemize}
        \item \textbf{Tables (Relations):} Store data in rows and columns
        \item \textbf{Rows (Tuples):} Individual records or entries
        \item \textbf{Columns (Attributes):} Fields or properties of data
        \item \textbf{Primary Keys:} Unique identifiers for records
        \item \textbf{Foreign Keys:} Link tables together
    \end{itemize}
\end{frame}

\begin{frame}{Example: Student Database}
    \begin{example}
        \centering
        \textbf{Students Table}
        
        \vspace{0.3cm}
        
        \begin{tabular}{|c|c|c|c|}
            \hline
            \textbf{Student\_ID} & \textbf{Name} & \textbf{Major} & \textbf{Year} \\
            \hline
            0152410071 & Samiha Mahzabin & Data Science & 2024 \\
            \hline
            0112410001 & Ahmed Khan & CS & 2024 \\
            \hline
            0152330002 & Sara Ali & Data Science & 2023 \\
            \hline
        \end{tabular}
    \end{example}
    
    \vspace{0.5cm}
    
    \begin{alertblock}{Note}
        Student\_ID serves as the \textbf{Primary Key}, uniquely identifying each student.
    \end{alertblock}
\end{frame}

\begin{frame}{Properties of Relational Databases}
    \begin{enumerate}
        \item \textbf{Data Independence:} Physical storage is separate from logical organization
        \item \textbf{Structured Query Language (SQL):} Standard language for querying and managing data
        \item \textbf{ACID Properties:}
        \begin{itemize}
            \item \textbf{A}tomicity: All or nothing transactions
            \item \textbf{C}onsistency: Data integrity maintained
            \item \textbf{I}solation: Concurrent transactions don't interfere
            \item \textbf{D}urability: Committed data is permanent
        \end{itemize}
        \item \textbf{Normalization:} Reduces data redundancy
    \end{enumerate}
\end{frame}

% Section 3
\section{Necessity of Database Management Systems}

\begin{frame}{Why Do We Need DBMS?}
    \begin{block}{Database Management System (DBMS)}
        Software that provides an interface between users/applications and the database, facilitating data creation, maintenance, and manipulation.
    \end{block}
    
    \vspace{0.5cm}
    
    \textbf{Problems without DBMS:}
    \begin{itemize}
        \item Data redundancy and inconsistency
        \item Difficulty in accessing data
        \item Data isolation
        \item Integrity problems
        \item Atomicity problems
        \item Concurrent access anomalies
        \item Security problems
    \end{itemize}
\end{frame}

\begin{frame}{Advantages of DBMS}
    \begin{columns}
        \column{0.5\textwidth}
        \textbf{Data Management:}
        \begin{itemize}
            \item Centralized control
            \item Reduced redundancy
            \item Data consistency
            \item Data integrity
        \end{itemize}
        
        \column{0.5\textwidth}
        \textbf{Operational Benefits:}
        \begin{itemize}
            \item Concurrent access
            \item Backup \& recovery
            \item Security controls
            \item Data independence
        \end{itemize}
    \end{columns}
    
    \vspace{0.5cm}
    
    \begin{exampleblock}{Real-world Impact}
        DBMS enables applications like online banking, e-commerce, social media, and enterprise resource planning systems to function efficiently and reliably.
    \end{exampleblock}
\end{frame}

\begin{frame}{DBMS vs File System}
    \centering
    \begin{tabular}{|p{4cm}|p{3cm}|p{3cm}|}
        \hline
        \textbf{Feature} & \textbf{File System} & \textbf{DBMS} \\
        \hline
        Data Redundancy & High & Minimal \\
        \hline
        Data Consistency & Poor & Excellent \\
        \hline
        Data Sharing & Limited & Extensive \\
        \hline
        Security & Basic & Advanced \\
        \hline
        Backup/Recovery & Manual & Automated \\
        \hline
        Query Language & None & SQL \\
        \hline
    \end{tabular}
\end{frame}

% Section 4
\section{Data Models}

\begin{frame}{What is a Data Model?}
    \begin{definition}
        A \textbf{data model} is a conceptual framework that defines how data is structured, stored, organized, and manipulated within a database system.
    \end{definition}
    
    \vspace{0.5cm}
    
    \begin{block}{Purpose of Data Models}
        \begin{itemize}
            \item Provide abstraction of real-world entities
            \item Define relationships between data elements
            \item Specify constraints and rules
            \item Guide database design and implementation
        \end{itemize}
    \end{block}
\end{frame}

\begin{frame}{Types of Data Models}
    \textbf{1. Hierarchical Data Model}
    \begin{itemize}
        \item Tree-like structure with parent-child relationships
        \item Example: File systems, organizational charts
    \end{itemize}
    
    \vspace{0.3cm}
    
    \textbf{2. Network Data Model}
    \begin{itemize}
        \item Graph structure with many-to-many relationships
        \item More flexible than hierarchical model
    \end{itemize}
    
    \vspace{0.3cm}
    
    \textbf{3. Relational Data Model}
    \begin{itemize}
        \item Data organized in tables (most popular)
        \item Based on mathematical set theory
    \end{itemize}
\end{frame}

\begin{frame}{Types of Data Models (Continued)}
    \textbf{4. Object-Oriented Data Model}
    \begin{itemize}
        \item Data represented as objects (like in OOP)
        \item Supports inheritance and encapsulation
    \end{itemize}
    
    \vspace{0.3cm}
    
    \textbf{5. NoSQL Data Models}
    \begin{itemize}
        \item \textbf{Document-based:} JSON/BSON documents (MongoDB)
        \item \textbf{Key-Value:} Simple key-value pairs (Redis)
        \item \textbf{Column-family:} Wide-column stores (Cassandra)
        \item \textbf{Graph:} Nodes and edges (Neo4j)
    \end{itemize}
    
    \vspace{0.3cm}
    
    \textbf{6. Entity-Relationship (ER) Model}
    \begin{itemize}
        \item Conceptual design tool
        \item Uses entities, attributes, and relationships
    \end{itemize}
\end{frame}

\begin{frame}{Data Model Abstraction Levels}
    \begin{enumerate}
        \item \textbf{Conceptual Level (External Schema)}
        \begin{itemize}
            \item User's view of the database
            \item What data users can access
        \end{itemize}
        
        \vspace{0.3cm}
        
        \item \textbf{Logical Level (Conceptual Schema)}
        \begin{itemize}
            \item Overall database structure
            \item Tables, relationships, constraints
        \end{itemize}
        
        \vspace{0.3cm}
        
        \item \textbf{Physical Level (Internal Schema)}
        \begin{itemize}
            \item How data is actually stored
            \item Indexes, file organization, storage
        \end{itemize}
    \end{enumerate}
\end{frame}

% Section 5
\section{Database Administration}

\begin{frame}{Role of Database Administrator (DBA)}
    \begin{block}{Database Administrator}
        A professional responsible for the design, implementation, maintenance, and security of an organization's databases.
    \end{block}
    
    \vspace{0.5cm}
    
    \textbf{Key Responsibilities:}
    \begin{itemize}
        \item Database design and implementation
        \item Performance monitoring and tuning
        \item Backup and recovery operations
        \item Security management and access control
        \item Database maintenance and upgrades
        \item Capacity planning
    \end{itemize}
\end{frame}

\begin{frame}{Database Administration Tasks}
    \begin{columns}
        \column{0.5\textwidth}
        \textbf{Planning \& Design:}
        \begin{itemize}
            \item Schema design
            \item Capacity planning
            \item Disaster recovery planning
        \end{itemize}
        
        \vspace{0.3cm}
        
        \textbf{Operations:}
        \begin{itemize}
            \item Performance tuning
            \item Query optimization
            \item Index management
        \end{itemize}
        
        \column{0.5\textwidth}
        \textbf{Security:}
        \begin{itemize}
            \item User authentication
            \item Access controls
            \item Encryption
        \end{itemize}
        
        \vspace{0.3cm}
        
        \textbf{Maintenance:}
        \begin{itemize}
            \item Backups
            \item Updates/patches
            \item Data archival
        \end{itemize}
    \end{columns}
\end{frame}

\begin{frame}{Database Security Considerations}
    \begin{alertblock}{Security Threats}
        \begin{itemize}
            \item Unauthorized access
            \item SQL injection attacks
            \item Data breaches
            \item Insider threats
        \end{itemize}
    \end{alertblock}
    
    \vspace{0.5cm}
    
    \textbf{Security Measures:}
    \begin{enumerate}
        \item Authentication and authorization
        \item Role-based access control (RBAC)
        \item Data encryption (at rest and in transit)
        \item Audit logging and monitoring
        \item Regular security assessments
        \item Backup encryption
    \end{enumerate}
\end{frame}

\begin{frame}{Database Performance Management}
    \textbf{Performance Optimization Techniques:}
    
    \begin{itemize}
        \item \textbf{Indexing:} Create appropriate indexes for faster queries
        \item \textbf{Query Optimization:} Rewrite inefficient queries
        \item \textbf{Database Tuning:} Adjust configuration parameters
        \item \textbf{Partitioning:} Split large tables for better performance
        \item \textbf{Caching:} Store frequently accessed data in memory
        \item \textbf{Regular Maintenance:} Update statistics, rebuild indexes
    \end{itemize}
    
    \vspace{0.5cm}
    
    \begin{exampleblock}{Monitoring}
        DBAs use monitoring tools to track CPU usage, memory consumption, disk I/O, and query performance to identify bottlenecks.
    \end{exampleblock}
\end{frame}

% Section 6
\section{Importance of DBMS}

\begin{frame}{Why DBMS Matters in Modern World}
    \textbf{Real-World Applications:}
    
    \begin{itemize}
        \item \textbf{Banking:} Account management, transactions, ATM systems
        \item \textbf{E-commerce:} Product catalogs, order processing, customer data
        \item \textbf{Healthcare:} Patient records, medical history, appointments
        \item \textbf{Education:} Student information systems, learning management
        \item \textbf{Social Media:} User profiles, posts, connections
        \item \textbf{Data Science:} Data warehouses, analytics, machine learning
    \end{itemize}
    
    \vspace{0.3cm}
    
    \begin{block}{For Data Science Students}
        Understanding DBMS is crucial for data extraction, transformation, and analysis in data science projects.
    \end{block}
\end{frame}

\begin{frame}{DBMS in Data Science}
    \textbf{Key Connections:}
    
    \begin{enumerate}
        \item \textbf{Data Collection \& Storage}
        \begin{itemize}
            \item Structured data storage for analysis
            \item Integration with data pipelines
        \end{itemize}
        
        \item \textbf{Data Preprocessing}
        \begin{itemize}
            \item SQL for data cleaning and transformation
            \item Joining multiple data sources
        \end{itemize}
        
        \item \textbf{Analytics \& Insights}
        \begin{itemize}
            \item Complex queries for pattern discovery
            \item Aggregations and statistical calculations
        \end{itemize}
        
        \item \textbf{Big Data Integration}
        \begin{itemize}
            \item NoSQL databases for unstructured data
            \item Data warehouses for analytics
        \end{itemize}
    \end{enumerate}
\end{frame}

\begin{frame}{Learning Outcomes Achieved}
    \begin{block}{Key Takeaways}
        After this presentation, you should understand:
    \end{block}
    
    \begin{itemize}
        \item \checkmark The fundamental concept of databases and their evolution
        \item \checkmark How relational databases organize and structure data
        \item \checkmark Why Database Management Systems are essential for modern applications
        \item \checkmark Different data models and when to use them
        \item \checkmark The critical role of database administration
        \item \checkmark The importance of DBMS in data science and various industries
    \end{itemize}
    
    \vspace{0.5cm}
    
    \begin{alertblock}{Remember}
        DBMS is the backbone of data-driven applications and essential for any data science professional!
    \end{alertblock}
\end{frame}

% Conclusion
\section{Conclusion}

\begin{frame}{Summary}
    \textbf{What We Covered:}
    
    \begin{itemize}
        \item Introduction to databases and their importance
        \item Relational database concepts and structure
        \item Necessity and advantages of DBMS
        \item Various data models for different use cases
        \item Database administration roles and responsibilities
        \item Real-world applications and relevance to data science
    \end{itemize}
    
    \vspace{0.5cm}
    
    \begin{center}
        \Large
        \textbf{Questions?}
    \end{center}
\end{frame}

\begin{frame}{References \& Further Reading}
    \textbf{Recommended Resources:}
    
    \begin{itemize}
        \item Database System Concepts - Silberschatz, Korth, Sudarshan
        \item Fundamentals of Database Systems - Elmasri \& Navathe
        \item SQL and Relational Theory - C.J. Date
        \item Online: W3Schools SQL Tutorial, PostgreSQL Documentation
    \end{itemize}
    
    \vspace{1cm}
    
    \begin{center}
        \Large
        \textbf{Thank You!}
        
        \vspace{0.5cm}
        
        \normalsize
        Nafiul Islam \\
        ID: 0152410070 \\
        Data Science, UIU
    \end{center}
\end{frame}

\end{document}
